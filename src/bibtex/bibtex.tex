\documentclass[mathserif]{beamer}

\usepackage{listings}
\usepackage{showexpl}
\usepackage{setspace}
\usepackage{multirow}
\usepackage{dtklogos}
\usepackage{caption}
\usepackage{url}
\usepackage{subcaption}
\usepackage[T1]{fontenc}

%\usepackage{handoutwithnotes}

%\pgfpagesuselayout{3 on 1 with notes}[a4paper,border shrink=5mm]

%\pgfpageslogicalpageoptions{1}{border code=\pgfusepath{stroke}}
%\pgfpageslogicalpageoptions{2}{border code=\pgfusepath{stroke}}
%\pgfpageslogicalpageoptions{3}{border code=\pgfusepath{stroke}}

\lstdefinestyle{latexsty}{
	language={[LaTeX]TeX},
    basicstyle=\small\ttfamily,
    breaklines=true,
    breakindent=0pt, 
    backgroundcolor=\color{lightgray},
    numbers=none, numberstyle=\tiny, stepnumber=1, numbersep=5pt,
    commentstyle=\color{red},
    showstringspaces=false,
    keywordstyle=\color{blue}\bfseries,
    morekeywords={align,begin},
    tabsize=2,
    pos=b
}

\usetheme{default}
\useoutertheme{infolines}
\usecolortheme[RGB={166,5,20}]{structure}
%\setbeamertemplate{items}[circle]
\setbeamertemplate{blocks}[rounded][shadow=false]
\setbeamertemplate{navigation symbols}{}
\setbeamertemplate{caption}[numbered]


\title{\LaTeX: An Introduction (Part 2)}
\subtitle{University Graduate College Training Course}
\author[Martin Chorley]{Dr Martin Chorley}
\institute[COMSC]{School of Computer Science \& Informatics, Cardiff University}
\date[21/02/14]{February 21st, 2014}


\begin{document}


%--------------- slide -------------------
\begin{frame}{Bibliographies in \LaTeX\ \& \BibTeX}
\vfill
\LaTeX\ includes basic support for Bibliographies without any extra packages or software. This is fine for relatively small documents with a small number of references.
\vfill
For more complicated documents (most papers and theses) you will want to use \BibTeX
\vfill
\end{frame}


%--------------- slide -------------------
\begin{frame}[fragile]
\frametitle{\LaTeX\ Bibliographies}
\vfill
The \texttt{thebibliography} environment allows us to specify a basic bibliography. It usually comes at the end of the document, before the \texttt{{\textbackslash}end\{document\}} command.
\vfill
\begin{LTXexample}[style=latexsty]
\begin{thebibliography}{9}

\bibitem{lamport94}
  Leslie Lamport,
  \emph{\LaTeX: A Document Preparation System}.
  Addison Wesley, Massachusetts,
  2nd Edition,
  1994.

\end{thebibliography}
\end{LTXexample}
\vfill
\end{frame}

%--------------- slide -------------------
\begin{frame}[fragile]
\frametitle{\LaTeX\ Bibliographies}
\vfill
Once we have a bibliography item we can cite it within our text with the \texttt{{\textbackslash}cite} command.
\vfill
\begin{lstlisting}[style=latexsty]
We refer to items from our bibliography using the cite command~\cite{lamport94}.
\end{lstlisting}
\vfill
We can refer to multiple items in the same citation:
\vfill
\begin{lstlisting}[style=latexsty]
\cite{lamport94, anotherref12, someref2013}
\end{lstlisting}
\vfill
\end{frame}


%--------------- slide -------------------
\begin{frame}[fragile]
\frametitle{\LaTeX\ Bibliographies - \texttt{natbib}}
\vfill
\LaTeX\ bibliography support can be extended simply using the \texttt{natbib} package.
\vfill
This allows you to specify the format of references and citations, and other options associated with bibliographies and referencing.
\vfill 
However, we won't be covering it today.
\vfill
\end{frame}

%--------------- slide -------------------
\begin{frame}[fragile]
\frametitle{\BibTeX}
\vfill
\BibTeX\ allows us to store our bibliographies as separate files (\texttt{.bib}), so keeping our \LaTeX\ source and bibliographic information separate.
\vfill
This allows us to re-use \texttt{.bib} files in different documents, or refer to items from many different bibliographies  within the same document.
\vfill
\end{frame}

%--------------- slide -------------------
\begin{frame}[fragile]
\frametitle{\BibTeX\ - data}
\vfill
\BibTeX\ files are plain text files in which the data representing the bibliography is structured.
\vfill
\begin{lstlisting}[style=latexsty]
@article{greenwade93,
    author  = "George D. Greenwade",
    title   = "The {C}omprehensive {T}ex {A}rchive {N}etwork ({CTAN})",
    year    = "1993",
    journal = "TUGBoat",
    volume  = "14",
    number  = "3",
    pages   = "342--351"
}
\end{lstlisting}
\vfill
\end{frame}


%--------------- slide -------------------
\begin{frame}[fragile]
\frametitle{\BibTeX\ - data}
\vfill
\BibTeX\ recognises many different types of references, including:
\vfill
\begin{center}
	\begin{tabular}{r | l }
		Type & Use \\
		\hline
		\texttt{article} & An article from a journal or magazine \\
		\texttt{book} & A book \\
		\texttt{conference} & The same as \texttt{inproceedings} \\
		\texttt{incollection} & Part of a book \\
		\texttt{inproceedings} & An article in a conference proceedings \\
		\texttt{manual} & Technical documentation \\		
		\texttt{mastersthesis} & A Masters degree thesis \\
		\texttt{misc} & For when nothing else is applicable \\
		\texttt{phdthesis} & A PhD degree thesis \\
		\texttt{proceedings} & The proceedings of a conference \\		
		\texttt{techreport} & A report published by an institution \\
		\texttt{unpublished} & A document that has not been formally published \\
	\end{tabular}
\end{center}
\vfill
\end{frame}

%--------------- slide -------------------
\begin{frame}[fragile]
\frametitle{\BibTeX\ - @article}
\vfill
An \texttt{@article} from a magazine or journal.
\vfill
\begin{description}
	\item[Required Fields] author, title, journal, year
	\item[Optional Fields] volume, number, pages, month, note
\end{description}
\vfill
\begin{lstlisting}[style=latexsty]
@article{bibtexkey,
    author  = "",
    title   = "",
    year    = "",
    journal = "",
}
\end{lstlisting}
\vfill
\end{frame}


%--------------- slide -------------------
\begin{frame}[fragile]
\frametitle{\BibTeX\ - @book}
\vfill
A \texttt{@book} that has been published.
\vfill
\begin{description}
	\item[Required Fields] author/editor, title, publisher, year
	\item[Optional Fields] volume/number, series, address, edition, month, note
\end{description}
\vfill
\begin{lstlisting}[style=latexsty]
@book{bibtexkey,
    author    = "",
    title     = "",
    publisher = "",
    year      = "",
}
\end{lstlisting}
\vfill
\end{frame}

%--------------- slide -------------------
\begin{frame}[fragile]
\frametitle{\BibTeX\ - @inproceedings}
\vfill
An article in a conference proceedings
\vfill
\begin{description}
	\item[Required Fields] author, title, booktitle, year
	\item[Optional Fields] editor, volume/number, series, pages, address, month, organization, publisher, note.
	\end{description}
\vfill
\begin{lstlisting}[style=latexsty]
@inproceedings{bibtexkey,
    author    = "",
    title     = "",
    booktitle = "",
    year      = "",
}
\end{lstlisting}
\vfill
\end{frame}

%--------------- slide -------------------
\begin{frame}[fragile]
\frametitle{\BibTeX\ - @techreport}
\vfill
A technical report from an institution (university, standardisation institute etc).
\vfill
\begin{description}
	\item[Required Fields] author, title, institution, year
	\item[Optional Fields] type, number, address, month, note
	\end{description}
\vfill
\begin{lstlisting}[style=latexsty]
@techreport{bibtexkey,
    author      = "",
    title       = "",
    institution = "",
    year        = "",
}
\end{lstlisting}
\vfill
\end{frame}

%--------------- slide -------------------
\begin{frame}[fragile]
\frametitle{\BibTeX\ - @misc}
\vfill
Anything that doesn't fit under any of the other types
\vfill
\begin{description}
	\item[Required Fields] none
	\item[Optional Fields] author, title, how published, month, year, note
	\end{description}
\vfill
\begin{lstlisting}[style=latexsty]
@techreport{bibtexkey,
    author       = "",
    title        = "",
    howpublished = "",
    year         = "",
}
\end{lstlisting}
\vfill
\end{frame}

%--------------- slide -------------------
\begin{frame}[fragile]
\frametitle{\BibTeX\ - Capitalisation}
\vfill
Occasionally, depending on the bibliography style being used, \BibTeX\ may remove capitalisation in titles of bibliography items.
\vfill
To preserve capitalisation, surround the items to be capitalised in curly braces.
\vfill
\begin{lstlisting}[style=latexsty]
    title = "The {LaTeX} Companion",
\end{lstlisting}
\vfill
or
\vfill
\begin{lstlisting}[style=latexsty]
    title = {{"The LaTeX Companion"}},
\end{lstlisting}
\vfill
\end{frame}

%--------------- slide -------------------
\begin{frame}[fragile]
\frametitle{\BibTeX\ - Authors}
\vfill
\BibTeX\ is able to deal with multiple authors for bibliographic items elegantly. 
\vfill
Names can be added either as \emph{Forename Surname} or \emph{Surname, Forename}.
\vfill
To add multiple authors, use 'and' between each author name
\begin{lstlisting}[style=latexsty]
	author = "Michel Goossens and Frank Mittelbach and Alexander Samarin"
\end{lstlisting}
\vfill
or
\vfill
\begin{lstlisting}[style=latexsty]
	author = "Goossens, Michel and Mittelbach, Frank and Samarin, Alexander"
\end{lstlisting}
\vfill
\end{frame}


%--------------- slide -------------------
\begin{frame}[fragile]
\frametitle{\BibTeX\ - Authors}
\vfill
The \emph{Surname, Forename} structure is more flexible in general.
\vfill
Last name prefixes (\emph{von}, \emph{van}, \emph{der}) are usually handled automatically, but if \BibTeX\ gets it wrong they can be grouped with the surname using curly braces.
\begin{lstlisting}[style=latexsty]
	author = "John {von Neumann}"
\end{lstlisting}
\vfill
or
\vfill
\begin{lstlisting}[style=latexsty]
	author = "von Neumann, John"
\end{lstlisting}
\vfill
\end{frame}


%--------------- slide -------------------
\begin{frame}[fragile]
\frametitle{\BibTeX\ - Urls}
\vfill
There is no \BibTeX\ field for URLs. To cite a website, you can use the \texttt{@misc} type and put the URL in the \texttt{howpublished} field. Alternatively, use the \texttt{note} field of \texttt{@techreport}, \texttt{@article}, or \texttt{@book}, 
\vfill
It is a good idea to use the \texttt{url} package and the \texttt{url} command when including URLs in \LaTeX\ to ensure proper formatting.
\vfill
\end{frame}


%--------------- slide -------------------
\begin{frame}[fragile]
\frametitle{Using \BibTeX}

\vfill
Once we have a \texttt{.bib} file full of bibliographic information, using it in \LaTeX\ is relatively straightforward.
\vfill
Throughout our document we use the \texttt{{\textbackslash}cite} command to reference items from our bibliography by their \BibTeX\ keys.
\vfill
We use the \texttt{{\textbackslash}bibliographystyle} command at the end of our document to let \BibTeX what style of reference we would like to use.
\vfill
We use the \texttt{{\textbackslash}bibliography} command to tell \BibTeX\ which file contains our bibliographic information
\vfill
\end{frame}

%--------------- slide -------------------
\begin{frame}[fragile]
\frametitle{Using \BibTeX}

\vfill
We then need to compile our documents a number of times in order to get references inserted in the correct place.
\begin{enumerate}
	\item We first run \texttt{latex}/\texttt{pdflatex} to find where there are missing references in the document
	\item We then run \texttt{bibtex} to extract the correct references from our \texttt{.bib} file.
	\item We then run \texttt{latex}/\texttt{pdflatex} to compile the document with the bibliography inserted
	\item We then run \texttt{latex}/\texttt{pdflatex} again to ensure all references are correct.
\end{enumerate}
\vfill

\end{frame}


%--------------- slide -------------------
\begin{frame}[fragile]
\frametitle{Using \BibTeX - Example}

\vfill
Demonstration of using \BibTeX\ within a document.
\vfill

\end{frame}

%--------------- slide -------------------
\begin{frame}[fragile]
\frametitle{\BibTeX\ Styles}

\vfill
There are many different bibliography styles which control how the references within the document and in the bibliography at the end of the document are presented.
\vfill
Which style to use will often be specified by a journal publisher/conference organiser.
\vfill

\end{frame}

%--------------- slide -------------------
\begin{frame}[fragile]
\frametitle{\BibTeX\ Tools}

\vfill
\BibTeX\ is a pretty powerful tool for handling referencing within \LaTeX\ documents.
\vfill
However, it can be annoying having to maintain \texttt{.bib} files for all your papers, your thesis, and so on.
\vfill
Fortunately, many tools will handle \texttt{.bib} files for you, automatically.
\vfill

\end{frame}

%--------------- slide -------------------
\begin{frame}[fragile]
\frametitle{\BibTeX\ Tools}

\vfill
Some suggestions:
\vfill
\begin{description}
	\item[Zotero] a web browser plugin for handling references that can output \texttt{.bib} files
	\item[Mendeley] A reference management application that can export \texttt{.bib} files
	\item[Google Scholar] can provide \BibTeX\ for any documents you find through it
\end{description}
\vfill

\end{frame}

%--------------- slide -------------------
\begin{frame}[fragile]
\frametitle{Exercise 3}
\vfill
Experiment with adding references into your document.
\vfill
Create and fill a \texttt{.bib} file.
\vfill
Explore \BibTeX\ styles and usage.
\vfill
\end{frame}


\end{document}